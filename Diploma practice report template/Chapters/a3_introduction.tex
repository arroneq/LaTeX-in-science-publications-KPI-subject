%!TEX root = ../thesis.tex
% створюємо вступ
У вступі ви коротенько (приблизно на сторінку) повинні окреслити 
актуальність вашого дослідження, тематику та проблематику, а також 
окреслити, яке саме завдання ви розв'язували при виконанні даного звіту. 
Без чіткої виразної постановки задачі дослідження, яка буде 
розв'язуватись на подальших сторінках, ніхто не зрозуміє, що тут 
відбувається, і до вас виникне багато, дуже багато запитань. Воно вам 
треба?

Зрозуміло, що звіт зазвичай присвячено огляду опублікованих результатів 
(якщо мова йде про грудневий звіт) або конкретно вашим результатам (якщо 
мова йде про переддипломну практику), однак це занадто загально. У вступі 
ви повинні окреслити, який саме огляд ви робите (наприклад: \emph{<<У даному звіті 
викладено результати новітніх методів сепулєнія на основі пост-квантово 
стійких сепулярізаторів, популярність яких була зумовлена...>>} і~т.д.), 
або які саме наукові результати збираєтесь одержати згідно ваших 
дослідницьких задач (які в майбутньому стануть задачами вашого 
бакалаврського чи магістерського диплому).