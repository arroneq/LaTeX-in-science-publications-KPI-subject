% !TeX program = lualatex
% !TeX encoding = utf8
% !TeX spellcheck = uk_UA

\documentclass[14pt]{extarticle}

\usepackage{fontspec}
\setmainfont{Times New Roman}
\usepackage[english, ukrainian]{babel}

\usepackage{geometry} 
\geometry{left=2cm}
\geometry{right=2cm}
\geometry{top=2cm}
\geometry{bottom=2cm}

\usepackage{indentfirst} % set an additional space before a paragraph at the begining of a new section

\usepackage[table,xcdraw,dvipsnames]{xcolor}
\usepackage{color}

\usepackage{hyperref}
\definecolor{linkcolor}{HTML}{0000FF}
\definecolor{urlcolor}{HTML}{0000FF} 
\hypersetup{
    pdfstartview=FitH, 
    unicode=true, 
    linkcolor=linkcolor, 
    urlcolor=urlcolor, colorlinks=true
}

\usepackage{enumitem}
\usepackage{amsmath, amssymb}
\usepackage{lipsum}

\usepackage{tabularray}
% https://tex.stackexchange.com/questions/642623/tabularray-correct-row-numbering-in-longtblr
\UseTblrLibrary{counter}

\NewDocumentCommand\NoArguments{}{Command with no arguments, just simple text to insert.}
\NewDocumentCommand\Addition{mmm}{#1 plus #2 equals #3}
\NewDocumentCommand\Circle{mmO{R}}{$#1^2 + #2^2 = #3^2$}

\begin{document}

\section*{Математична верстка тексту}

I'm here! ґ ї The simplest type of macro is one with no arguments at all. This isn’t going to show off xparse very much but is a starting point. The traditional method to do this is: 

\begin{enumerate}[label=($\bigstar$)]
    \item \NoArguments
    \item Let us add some numbers: \Addition{1}{2}{3}, and then \Addition{5}{6}{11}.
    \item Another example: \Circle{a}{b}[c] and \Circle{x}{y} are the same.
\end{enumerate}

Я пишу в \LaTeX{} різноманітні формули, наприклад: \Circle{x}{y}. Пізніше я ще напишу купу всякого. Ось, наприклад формула між абзацами:
\begin{equation}\label{eq: circle equation}
    x^2 + y^2 = R^2
\end{equation}

Наостанок, формула \eqref{eq: circle equation} -- \& це рівняння кола. Захищаючи Україну зараз, ми па\-м'ятаємо, що наша міцність і сила духу базуються на міцності й силі багатьох українців, які не здавалися, які мріяли та діяли заради того, щоб Україна жила.

\section*{Оформлення таблиці}

Перш ніж розглядати коротку таблицю в оточенні \texttt{tblr}, проілюструємо усі переваги використання іншого оточення, яке має назву \texttt{longtblr}: 

\DefTblrTemplate{contfoot-text}{default}{\itshape\footnotesize Продовження на наступній сторінці}
\DefTblrTemplate{conthead-text}{default}{(продовження)}

\newcounter{number}\setcounter{number}{0}
\begin{longtblr}[
        caption={Довга таблиця}, 
        label={table: long table},
    ]{
        hlines={1pt, solid}, hline{1,2,Z}={2pt, solid},
        vlines={1pt}, vline{1,2,Z}={2pt, solid},
        colspec={cXXXX}, % X - розтягнення колонки на доступну ширину сторінки
        cell{1}{2}={r=1, c=4}{c},
        column{2,3,4,5}={mode=math, c},
        row{1}={mode=text, bg=purple7, fg=white, font=\bfseries},
        row{even}={bg=gray9},
    }
    
    \textnumero                            & Таблиця множення &               &                &               \\
    \refstepcounter{number}\arabic{number} & 2\times 2=4      &  3\times 3=9  & 4\times 4=16   &  5\times 5=25 \\
    \refstepcounter{number}\arabic{number} & 2\times 2=4      &  3\times 3=9  & 4\times 4=16   &  5\times 5=25 \\
    \refstepcounter{number}\arabic{number} & 2\times 2=4      &  3\times 3=9  & 4\times 4=16   &  5\times 5=25 \\
    \refstepcounter{number}\arabic{number} & 2\times 2=4      &  3\times 3=9  & 4\times 4=16   &  5\times 5=25 \\
    \refstepcounter{number}\arabic{number} & 2\times 2=4      &  3\times 3=9  & 4\times 4=16   &  5\times 5=25 \\
    \refstepcounter{number}\arabic{number} & 2\times 2=4      &  3\times 3=9  & 4\times 4=16   &  5\times 5=25 \\
    \refstepcounter{number}\arabic{number} & 2\times 2=4      &  3\times 3=9  & 4\times 4=16   &  5\times 5=25 \\
    \refstepcounter{number}\arabic{number} & 2\times 2=4      &  3\times 3=9  & 4\times 4=16   &  5\times 5=25 \\
    \refstepcounter{number}\arabic{number} & 2\times 2=4      &  3\times 3=9  & 4\times 4=16   &  5\times 5=25 \\
    \refstepcounter{number}\arabic{number} & 2\times 2=4      &  3\times 3=9  & 4\times 4=16   &  5\times 5=25 \\
    \refstepcounter{number}\arabic{number} & 2\times 2=4      &  3\times 3=9  & 4\times 4=16   &  5\times 5=25 \\
    \refstepcounter{number}\arabic{number} & 2\times 2=4      &  3\times 3=9  & 4\times 4=16   &  5\times 5=25 \\
    \refstepcounter{number}\arabic{number} & 2\times 2=4      &  3\times 3=9  & 4\times 4=16   &  5\times 5=25 \\
    \refstepcounter{number}\arabic{number} & 2\times 2=4      &  3\times 3=9  & 4\times 4=16   &  5\times 5=25 \\
\end{longtblr}

\vspace{0.4cm}
В порівнянні, оточення \texttt{tblr} має свої особливості. Наприклад, дещо незграбно відображається заголовок над таблицею --- як результат, назву таблиці було зсунуто вниз.

\begin{table}[h!]\centering
    \begin{tblr}{
            hlines={1pt,solid}, vlines={1pt,solid},
            colspec={cc},
            cell{3}{1-Z}={wd=6cm},
            cell{1}{1}={r=1, c=2}{c},
            row{3}={valign={m}, font=\itshape}
        }
        
        Холодні пори року & \\
        Зима & Осінь \\
        Українська зима порівняно м'яка, з частими відлигами. Проте інколи бувають сильні снігопади, хуртовини, ожеледь, тумани, у горах можуть сходити снігові лавини 
        & Загалом весна є наслідком приходу сонячного тепла, однак на погоду впливають також інші, менш передбачувані, явища. Навесні погода буває особливо мінливою: з частими похолоданнями, вітрами, заморозками, різкими підняттями температури тощо \\
    \end{tblr}
    \caption{Коротка таблиця}
    \label{table: short table}
\end{table}

\end{document}