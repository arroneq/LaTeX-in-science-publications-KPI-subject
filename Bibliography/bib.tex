% !TeX program = lualatex
% !TeX encoding = utf8
% !BIB program = biber

\documentclass[14pt]{extarticle}

% bibstyles https://www.google.com/url?sa=t&rct=j&q=&esrc=s&source=web&cd=&cad=rja&uact=8&ved=2ahUKEwj0zY3pkMr9AhUK_CoKHXAlDTwQFnoECAkQAQ&url=http%3A%2F%2Ftug.ctan.org%2Finfo%2Fbiblatex-cheatsheet%2Fbiblatex-cheatsheet.pdf&usg=AOvVaw2QMCRt2AGY8VMmFGgM17aS
\usepackage[bibstyle=gost-numeric]{biblatex}
\addbibresource{bibliography.bib}
\usepackage[autostyle=false]{csquotes}

\usepackage{fontspec}
% \setmainfont{CMU Serif}
\setmainfont{Times New Roman}
\usepackage[english, ukrainian]{babel}

\usepackage{geometry} 
\geometry{left=2cm}
\geometry{right=2cm}
\geometry{top=2cm}
\geometry{bottom=2cm}

\usepackage{enumitem}
\usepackage{lipsum}

% \usepackage{tabularray}

\usepackage{xcolor}

\usepackage{hyperref}
\definecolor{linkcolor}{HTML}{0000FF}
\definecolor{urlcolor}{HTML}{0000FF} 
\definecolor{citecolor}{HTML}{0000FF} 
\hypersetup{
    pdfstartview=FitH, 
    unicode=true, 
    linkcolor=linkcolor, 
    urlcolor=urlcolor,
    citecolor=citecolor, 
    colorlinks=true
}

\usepackage{amsmath, amssymb}

% \NewDocumentCommand: https://www.texdev.net/2010/05/23/from-newcommand-to-newdocumentcommand/
\NewDocumentCommand\NoArguments{}{Command with no arguments, just simple text to insert.}
\NewDocumentCommand\Addition{mmm}{#1 plus #2 equals #3}
\NewDocumentCommand\Circle{mmO{R}}{$#1^2 + #2^2 = #3^2$}

\begin{document}

I'm here! The simplest type of macro is one with no arguments at all. This isn’t going to show off xparse very much but is a starting point. The traditional method to do this is: 

\begin{enumerate}[label=($\bigstar$)]
    \item \NoArguments
    \item Let us add some numbers: \Addition{1}{2}{3}, and then \Addition{5}{6}{11}.
    \item Another example: \Circle{a}{b}[c] and \Circle{x}{y} are the same.
\end{enumerate}

Я пишу в \LaTeX{} різноманітні формули, наприклад: \Circle{x}{y}. Пізніше я ще напишу купу всякого. Ось, наприклад формула між абзацами:
\begin{equation}\label{eq: circle equation}
    x^2 + y^2 = R^2
\end{equation}

\newcounter{par}
\newcommand{\parcount}{--\hspace{0.1cm}\addtocounter{par}{1}\thepar\hspace{0.1cm}--\hspace{0.2cm}}

\parcount \lipsum[2]

\parcount \lipsum[3]

\parcount Наостанок, формула \eqref{eq: circle equation} -- це рівняння кола.Захищаючи Україну зараз, ми пам'ятаємо, що наша міцність і \cite{Gentle1998} сила духу базуються на міцності й силі багатьох українців, які не здавалися, які мріяли та діяли заради того, щоб Україна жила. Самостійна, \cite{Eddy2004} вільна і сильна Україна, європейська і демократична, \cite{Sharma2009} незалежна і цілісна.

\newpage
\printbibliography

\end{document}