%!TEX root = ../thesis.tex
% створюємо вступ
\textbf{Актуальність дослідження.} Актуальність даного дослідження полягає 
у тому, що без нього ви не одержите диплом про вищу освіту. Відповідно, ви повинні 
оформити результати вашого дослідження належним чином.

Вступ є однією із самих формалізованих частин дипломної роботи. На початку 
ви у двох-трьох абзацах повинні окреслити проблематику та актуальність 
вашого дослідження, після чого переходити до мети та завдання.

\textbf{Метою дослідження} є певна абстрактна недосяжна річ на кшталт 
загальнолюдського щастя на горизонті. Для досягнення мети необхідно 
розв'язати \textbf{задачу дослідження}, яка полягає у чомусь суттєво більш 
конкретному. Для розв'язання задачі необхідно вирішити такі завдання:

\begin{enumerate}
\item провести огляд опублікованих джерел за тематикою дослідження;
\item (наступний пункт, пов'язаний із теоретичним дослідженням);
\item (і ще один, наприклад, про експериментальну перевірку результатів);
\item (і взагалі, краще із науковим керівником проконсультуйтесь, як ваші 
завдання правильно писати).
\end{enumerate}

\emph{Об'єктом дослідження} є якісь процеси або явища загального 
характеру (наприклад, <<інформаційні процеси в системах криптографічного 
захисту>>).

\emph{Предметом дослідження} є конкретний математичний чи фізичний 
об'єкт, який розглядається у вашій роботі та який можна трактувати
як певну властивість об'єкта дослідження (наприклад, <<моделі та методи
диференціального криптоаналізу ітеративних симетричних блочних шифрів>>).

При розв’язанні поставлених завдань використовувались такі \emph{методи дослідження}: і 
тут коротенький перелік (наприклад, але не обмежуючись: методи лінійної та абстрактної 
алгебри, теорії імовірностей, математичної статистики, комбінаторного 
аналізу, теорії кодування, теорії складності алгоритмів, методи 
комп’ютерного та статистичного моделювання) 

\textbf{Наукова новизна} отриманих результатів полягає... -- тут необхідно 
перелічити, що саме нового з точки зору науки несе ваша робота. До усіх 
тверджень, які сюди виносяться, подумки (а іноді й явним чином) потрібно 
ставити слово <<вперше>> -- і ці твердження повинні залишатись істинними.

\textbf{Практичне значення} результатів полягає... -- тут необхідно 
зазначити практичну користь від результатів вашої роботи. Що саме можна 
покращити, підвищити (або знизити), зробити гарного (або уникнути 
поганого) після вашого дослідження.

\textbf{Апробація результатів та публікації.} Наприкінці вступу необхідно 
зазначити перелік конференцій, семінарів та публікацій, в яких викладено 
результати вашої роботи. Якщо результати вашої роботи ніде не 
доповідались, опускайте даний абзац.